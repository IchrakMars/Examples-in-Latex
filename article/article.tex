% MARINA VON STEINKIRCH
% http://mysbfiles.stonybrook.edu/~mvonsteinkir/

\documentclass[10pt]{article} % Good sizes are usually 10pt or 11pt. You can use [10pt,twocolumn] to generate a document with two columns.
\title{An Article in Latex for Physics}
\author{ \texttt{ Marina von Steinkirch, steinkirch@gmail.com}\\
	   \texttt{State University of New York at Stony Brook}}
\date{\today} % Or type the date you want




%%%%%%%%%%%%%%%   PACKAGES   %%%%%%%%%%%%%%%%%%%%%%%%%%%

\usepackage{amsmath}    % Package  for subequations
\usepackage{graphicx}   % Package for figures
\usepackage{verbatim}   % Package  for program listings
\usepackage{color}      % Package to insert  color in text
\usepackage{hyperref}   % Package  for hypertext links, external documents and URLs
\usepackage{amssymb} % For mathematical constructions
\usepackage{latexsym} % Package to generate mathematical symbols
\usepackage{makeidx} % Package to generate an index in the end

\makeindex % To Generate an index in the end





%%%%%%%%%%%%%%%  BEGIN DOCUMENT   %%%%%%% %%%%%%%%%%%%%%%%%%%%%%%

\begin{document}
\maketitle
\numberwithin{equation}{section} % To give the number of each equation related to the section
\tableofcontents % to print the index

\begin{abstract}
A very brief introduction to articles in latex.
\end{abstract}





%%%%%%%%%%%%%%    BODY    %%%%%%%%%%%%%%%%%%%%%%%%%%%%%%%%%%%%%


\section{Some Basic Code} % You can write \section* to have it without numeration

\subsection{Basic Commands}

\begin{itemize}
\item This is how you include footnotes \footnote{My footnote}
\item To make a reference to some item in the bibliography, use \cite{steinkirch}. To make a reference to an equation, use \eqref{myequation}. To make a reference to a table or figure, use \ref{mytable}.
\item If you want to index topics of your text, you can use \index{How to make an idex}.
\end{itemize}

\subsubsection*{Font Styles}


\begin{itemize}
\item To write in bold, {\bf writing in bold }
\item To write in italic, {\it writing in bold }
\item To write in typing letters, {\texttt typing letters}
\item To write in small {\small small text}, or very small {\tiny tiny text}.
\item To start a quotation \begin{quotation} Quoting mamma... \end{quotation}
\item To use calligraph letters, use $\mathcal{L}$.
\end{itemize}



\subsection{Declaring Equations} % You can write \subsection* to have it without numeration

You can include equations in the middle of the tex using $2\pi$.

Equations in different lines without numerations can be written as 
$$1.19 \times 10^{57}$$ 

Equations with numeration can be declared as
\begin{equation} % however, if you write equation*  you lose the numeration
P \Big (  \frac{nRT}{P} \Big ) = 1.
\label{myequation} % labeling equations to make references in the text
\end{equation}

or when you have an array of equations,
\begin{eqnarray} % however, if you write eqnarray*  you lose the numeration
P \Big (  \frac{nRT}{P} \Big ) &=& 1 + k,\\  % The symbol & makes the indentation 
&=& 0. 
\end{eqnarray}

Multi-line equations can be written as
\begin{equation}
\label{myequation_in_many_lines}
\begin{split}
a& = b+c\\
& \quad + d - e\\
& =1
\end{split}
\end{equation}



\subsection{Including Tables, Lists, and Figures} 


This is the code for including tables:
\begin{table}[htdp]
\begin{center}
\begin{tabular}{c|c} % This parameter says how many columns and their respective lines, another example {|cccc|}
\hline % Vertical lines
Testing 1 & Testing 2\\
Testing 4 & Testing 3 \\
\hline % Vertical lines
\end{tabular}
\end{center}
\caption{Data for the Problem.} 
\label{mytable}
\end{table}


This is the code for including lists:
\begin{itemize}
\item Item one
\item Item two
\end{itemize}

This is the code for including enumerated lists:
\begin{enumerate}
\item First Item
\item Second Item
\end{enumerate}

This is the code for including figures:
\begin{figure}[htbp]
\begin{center}
%\includegraphics[scale=0.5]{fig.jpg} % you can set the scale or length and width
\label{myfigure}
\end{center}
\end{figure}

\section{The End of The Document}
%%%%%%%%%%%%%%%%%%%%%%%%%  END OF DOCUMENT %%%%%%%%%%%%%%%%%%%%%%%%


\begin{thebibliography}{2} % Declaration of bibliography
\bibitem{atiyah} ATIYAH,  The Geometry and Physics of Knots,1990.
\bibitem{steinkirch} Marina Von Steinkirch,  Templates in Latex, \url{http://mysbfiles.stonybrook.edu/~mvonsteinkir}.
\end{thebibliography}

\printindex % Print the index of words that you had indexed by \index{my-indexation}.

\end{document}

